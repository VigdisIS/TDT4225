\section{Theory}

\begin{enumerate}
    \item 
        SSD 
    \item 
        asd
    \item 
        sad
    \item 
        asdasd
    \item 
        The numbers start at 1 with each use of the \texttt{enumerate} environment.
    \item 
        Another entry in the list
    \item 
        When it comes to fault tolerance, a hardware error could be a disk, 
        power, CPU, or network failure. A software error could be a bug in the
        system causing errors. It could also be system crashes due to errors 
        caused by the system having been turned on for too long. Human errors 
        are usually due to accidental actions such as deleting or overwriting a 
        file by mistake, or inserting wrong data. 
    \item 
        To achieve fault tolerance, one could restart computers frequently to
        avoid crashes due to the system having been running for too long. One 
        could also store the data across multiple computers such that in the 
        case one computer fails, only a small portion of the data is affected, 
        though this is hard to do on stateful systems. Another way to achieve
        fault tolerance is by replication, i.e. having the same data stored on
        multiple computers. This way, if one computer fails, the data is still
        available on the other computers, though this method could be expensive.
    \item
        SQL is a relational database where relationships between different data
        in the database can be easily defined using joins. In a document 
        database, however, this is not the case. This is due to the fact that 
        the data is stored in documents, and the documents are self-contained,
        i.e. not split into multiple tables. This structure also makes it bad 
        for supporting many-to-many relationships, something SQL is good at. In
        the case where we wish to model a paper that has many sections and 
        words, and additionally many authors, where each author with name and 
        address have many written papers, we would have to store each entity
        in its own self-contained document. So we would have one document
        representing each paper with sections and words, in addition to the 
        authors of the paper. We would have another document representing all 
        authors with their name and address. The relationship defining which
        authors have written which papers would be stored in the paper document,
        though to extract their name and address we would have to define 
        many-to-many relationshops not contained in any of the documents,
        presumably using application code or some other method. Using only
        self-contained documents, we would have to store a lot of redundant, 
        duplicate data where each paper has all authors, with their names and 
        adresses as well, listed in the document. This is however not a good 
        solution due to the fact that in different papers, the same author could
        have their name spelt differently or have some name changes. Trying
        to get all papers written by a specific author could prove difficult
        due to this.
    \item 
        If the data comes in the form of self-contained documents where there 
        are few relationshops between them, document databases are a good fit.
        However, in the case where anything could be related to everything, a 
        graph database would be a better fit. For example, when we want to store
        data containing people and how they are related to each other, a graph
        database would be preferable over a document database, as the data is
        highly relational.
    \item 
        asdasd 
    \item 
        asdasd
    \item
        sad
  \end{enumerate}
