\section{Theory}

\begin{enumerate}
    \item 
        In an SSD, the Flash Translation Layer (FTL) is responsible for wear 
        leveling, meaning distributing writes evenly across the entire disk. The 
        data in the NAND, the type of flash memory used by most SSDs, cannot be 
        updated in-place, only by erase-and-write cycles, which each NAND only 
        has a limited number of. Furthermore, since in an SSD whole blocks have 
        to be deleted when we wish to delete only a single page, we have to 
        rewrite the entire block, except the page we wish to delete, to a new 
        block. Frequent updates like this to a set of blocks will wear the 
        blocks holding these pages out much faster than the other blocks. 
        Therefore, writing to all blocks uniformly over time means no block is 
        more likely to fail than others, and we can prolong the lifespan of the
        SSD.
    \item 
        Sequential writes are important for performance on SSDs as when the disk
        reaches full capacity, garbage collection has to be performed. This
        becomes a very complex task when the data is written randomly, due to 
        the FTL having to always maintain the mapping between logical and 
        physical block addresses across the disk. When the data is written 
        sequentially, however, invalidating blocks as eligible for erasure is 
        done block by block since thats how they're written, and the write 
        amplification is greatly reduced.
        
    \item 
        Alignment of blocks to clustered SSD pages improves performance by 
        lowering the write latency in an SSD due to the fact that the write 
        requests can be written to disk without any further write overhead. If 
        the write request is not aligned to the page size, however, the SSD has 
        to read the rest of the content of the previous clustered page, and then 
        subsequently merge this with the updated data before being able to write 
        the entire page back to the disk.
    \item 
        The layout of a MemTable in RocksDB follows the structure where updates
        are first inserted into the memtable until it reaches its maximum size.
        RocksDB will then subsequently generate a new memtable for new updates,
        and the old memtable's entries will be flushed to disk as an SSTable.

        The layout of an SSTable in RocksDB follows a block-based table format 
        where key-value pairs are stored in sorted order in blocks, along with
        metadata and indexes.
    \item 
        During compaction in RocksDB, multiple SStables from the same level in
        the LSM-tree are merged into a single SSTable in the next level, which 
        are together then merged into a new, even larger SStable. If a key has
        been marked as deleted during this process, it will be discarded and 
        only the most recent version of a key is kept. This newly generated 
        SSTable is then moved to the next level, and if this causes the next
        level to have too many SSTables, this whole process is repeated in this
        level.
    \item 
        LSM-trees are regarded as more efficient than B+-trees for large volumes
        of inserts due to the fact that the the B+-tree has to write to disk 
        every time a new entry is inserted into the tree. LSM-trees on the other
        hand buffer writes in their memtables, and only flushing to disk when
        the memtable is full.
    \item 
        When it comes to fault tolerance, a hardware error could be a disk, 
        power, CPU, or network failure. A software error could be a bug in the
        system causing errors. It could also be system crashes due to errors 
        caused by the system having been turned on for too long. Human errors 
        are usually due to accidental actions such as deleting or overwriting a 
        file by mistake, or inserting wrong data. 
    \item 
        To achieve fault tolerance, one could restart computers frequently to
        avoid crashes due to the system having been running for too long. One 
        could also store the data across multiple computers such that in the 
        case one computer fails, only a small portion of the data is affected, 
        though this is hard to do on stateful systems. Another way to achieve
        fault tolerance is by replication, i.e. having the same data stored on
        multiple computers. This way, if one computer fails, the data is still
        available on the other computers, though this method could be expensive.
    \item
        SQL is a relational database where relationships between different data
        in the database can be easily defined using joins. In a document 
        database, however, this is not the case. This is due to the fact that 
        the data is stored in documents, and the documents are self-contained,
        i.e. not split into multiple tables. This structure also makes it bad 
        for supporting many-to-many relationships, something SQL is good at. In
        the case where we wish to model a paper that has many sections and 
        words, and additionally many authors, where each author with name and 
        address have many written papers, we would have to store each entity
        in its own self-contained document. So we would have one document
        representing each paper with sections and words, in addition to the 
        authors of the paper. We would have another document representing all 
        authors with their name and address. The relationship defining which
        authors have written which papers would be stored in the paper document,
        though to extract their name and address we would have to define 
        many-to-many relationshops not contained in any of the documents,
        presumably using application code or some other method. Using only
        self-contained documents, we would have to store a lot of redundant, 
        duplicate data where each paper has all authors, with their names and 
        adresses as well, listed in the document. This is however not a good 
        solution due to the fact that in different papers, the same author could
        have their name spelt differently or have some name changes. Trying
        to get all papers written by a specific author could prove difficult
        due to this.
    \item 
        If the data comes in the form of self-contained documents where there 
        are few relationshops between them, document databases are a good fit.
        However, in the case where anything could be related to everything, a 
        graph database would be a better fit. For example, when we want to store
        data containing people and how they are related to each other, a graph
        database would be preferable over a document database, as the data is
        highly relational.
    \item 
        asdasd 
    \item 
        asdasd
    \item
        sad
  \end{enumerate}
