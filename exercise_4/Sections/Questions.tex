\section{Questions}

\begin{enumerate}
    \item 
        \begin{enumerate}
            \item One should use multi-leader replication in cases where 
                \begin{enumerate} 
                    \item we have multiple datacenters
                    \item we have an application that needs to continue working
                        even in cases where its connection to the internet 
                        has dropped.
                    \item we have a system in which multiple users need to be 
                        able to collaborate and edit in real time.
                \end{enumerate}
                This type of replication allows for better performance, since 
                writes can be processed locally, and then replicated 
                asynchronously to other datacenters. Additionally, this makes 
                for a higher tolerance of outages since each datacenter can 
                continue operating normally even when one datacenter has failed. 
                A replication like this will also make it so that the system
                tolerates network problems better, since a temporary network
                outage would not prevent writes from being processed.
                \\ One could use leader-based replication in single datacenters or
                in general in cases where for instance it is not a problem for 
                the system that all writes must go through this one leader, and 
                that the system won't often experience network interruption
                \cite[pp.~168-170]{kleppmann_2017}.
            \item One should use log shipping as a replication means instead of
                just replicating the SQL statements because sending SQL
                statements can be unpredictable in terms of non-deterministic
                functions, where functions calling for a time now or a randomly
                generated number will be different on the different replicas. 
                This method can also cause unpredictable scenarios in terms of
                auto-incrementing IDs, in addition to side effects
                possibly having different results on different replicas 
                \cite[p.~159]{kleppmann_2017}.
        \end{enumerate}
  \end{enumerate}
